\documentclass[11pt, a4paper, titlepage, headings=standardclasses]{scrartcl}

\usepackage[margin=2.5cm]{geometry}

\author{Max Mihailescu}
\title{Technical Student am CERN}
\subtitle{Praktikumsbericht}
\date{01. Oktober 2019 -- 31. August 2020}

\usepackage[backend=biber, url=false]{biblatex} 
\bibliography{bibliography.bib}

\begin{document}
\maketitle
\KOMAoptions{titlepage=false}

\begin{abstract}
	\subsection*{Zusammenfassung}
	\noindent Vom 1. Oktober 2019 bis zum 31. August 2020 habe ich am CERN ein Praktikum als Technical Student absolviert. Dieses Programm richtet sich an Studierende verschiedener Studiengänge und hat das Ziel, dass die Studierenden praktische Erfahrungen in einem wissenschaftlichen Umfeld sammeln können.
	
	Ich habe die elf Monate im Beams Department verbracht, und dort in der Sektion die sich mit den Teilchenquellen beschäftigt und somit den Anfang der Beschleunigerkette bedient. Hier habe ich an der Quelle für Bleiionen mitgearbeitet und meine Aufgabe war mittels Techniken des Machine Learnings die Quellenbetrieb durch die Operatoren zu unterstützen und zu erleichtern.
\end{abstract}

\section{Vorstellung der Organisation}
Das CERN ist eine europäisches Labor mit Sitz in Meyrin bei Genf, das sich mit der Erforschung von Teilchenphysik beschäftigt. Gegründet wurde die Organisation 1954 und hat mittlerweile 23 Mitgliedsstaaten, sowie 8 assoziierte Mitgliedsstaaten \cite{CERN:AnnualReport2019}.

\section{Aufgaben als Praktikant}


\section{Methodik}


\section{Fazit}

\printbibliography
\end{document}